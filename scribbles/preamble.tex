% SPDX-FileCopyrightText: 2022 Alexandru Fikl <alexfikl@gmail.com>
% SPDX-License-Identifier: CC0-1.0

% <<< packages

\usepackage[utf8]{inputenc}
\usepackage[activate={true,nocompatibility},
            final,
            tracking=true,
            kerning=true,
            spacing=true,
            factor=1100,
            stretch=10,
            shrink=10]{microtype}
\microtypecontext{spacing=nonfrench}

\usepackage{xparse}
\usepackage{scrlayer-scrpage}

\usepackage{fixmath}
\usepackage{amsmath}
\usepackage{amsthm}
\usepackage{amssymb}
\usepackage{stmaryrd}
\usepackage{mathrsfs}
\usepackage{xfrac}

\usepackage{tikz}
\usepackage[mathlines]{lineno}
\usepackage{booktabs}
\usepackage[shortlabels]{enumitem}

\usepackage{hyperref}
\usepackage{cleveref}

\usepackage{lmodern}

% >>>

% <<< formatting

% set default page margins
\makeatletter
\@ifundefined{KOMAoptions}{}{
\KOMAoptions{
    DIV=14,
    parskip=half*,
    overfullrule}}

% custom headers and footers
\pagestyle{plain}
\clearscrheadfoot
\refoot[\pagemark]{\pagemark}
\rofoot[\pagemark]{\pagemark}

% less jarring links
\hypersetup{
    colorlinks=true,
    urlcolor=blue,
    citecolor=black,
    linkcolor=black
}

% simple todo command
\usepackage{todonotes}
\newcommand{\unsure}[1]{
    \todo[linecolor=red,backgroundcolor=red!25,bordercolor=red,inline]{#1}}

% add line numbers
\renewcommand{\linenumberfont}{
    \normalfont\footnotesize\textcolor{black!50}
}

\linenumbers

% number equations within sections
\numberwithin{equation}{section}

% >>>

% <<< commands

% <<< math commands

% integral element with proper spacing
\NewDocumentCommand \dx { O{x} } {\,\mathrm{d} #1}
% style for displaying vectors
\NewDocumentCommand \vect { m } { \mathbold{#1} }

% jump notation
\NewDocumentCommand \jump { sm } {
    \IfBooleanTF#1
    {\left\llbracket #2 \right\rrbracket}
    {\llbracket #2 \rrbracket}
}
% average notation
\NewDocumentCommand \avg { sm } {
    \IfBooleanTF#1
    {\left\langle #2 \right\rangle}
    {\langle #2 \rangle}
}
% inner product
\NewDocumentCommand \ip { m } { \avg{ #1 } }

% ordinary derivative
\NewDocumentCommand \od { m m } { \dfrac{\mathrm{d} #1}{\mathrm{d} #2} }
% material derivative
\NewDocumentCommand \md { m m } { \dfrac{\mathrm{D} #1}{\mathrm{D} #2} }
% partial derivative
\NewDocumentCommand \pd { m m } { \dfrac{\partial #1}{\partial #2} }

% set of real number
\NewDocumentCommand \R {} { \mathbb{R} }

% >>>

% <<< math operators

\DeclareMathOperator{\tr}{tr}
\DeclareMathOperator{\sech}{sech}
\DeclareMathOperator{\argmin}{\operatorname{arg}\,\operatorname{min}}
\DeclareMathOperator{\esssup}{\operatorname{ess}\,\operatorname{sup}}

% >>>

% <<< theorem environments

\newtheorem{theorem}{Theorem}
\newtheorem{remark}{Remark}
\newcommand{\remarkautorefname}{Remark}
\newtheorem{lemma}{Lemma}
\newcommand{\lemmaautorefname}{Lemma}
\newtheorem{example}{Example}
\newcommand{\exampleautorefname}{Example}
\newtheorem{proposition}{Proposition}
\newcommand{\propositionautorefname}{Proposition}
\newtheorem{definition}{Definition}
\newcommand{\definitionautorefname}{Definition}

% >>>

% >>>

% <<< title

\newcommand{\horrule}[1]{\rule{\linewidth}{#1}}
\makeatletter
\def\@maketitle{
\begin{center}
    \normalfont \normalsize

    \horrule{0.5pt}
    \vspace{0.2cm}

    {\huge \bfseries \@title \par}

    \vspace{0.3cm}
    \horrule{2pt}
\end{center}
}
\makeatother

% >>>

% vim:foldmarker=<<<,>>>:foldmethod=marker:nospell
