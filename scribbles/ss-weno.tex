\documentclass{scrartcl}
% SPDX-FileCopyrightText: 2022 Alexandru Fikl <alexfikl@gmail.com>
% SPDX-License-Identifier: CC0-1.0

% <<< packages

\usepackage[utf8]{inputenc}
\usepackage[activate={true,nocompatibility},
            final,
            tracking=true,
            kerning=true,
            spacing=true,
            factor=1100,
            stretch=10,
            shrink=10]{microtype}
\microtypecontext{spacing=nonfrench}

\usepackage{xparse}
\usepackage{scrlayer-scrpage}

\usepackage{fixmath}
\usepackage{amsmath}
\usepackage{amsthm}
\usepackage{amssymb}
\usepackage{stmaryrd}
\usepackage{mathrsfs}
\usepackage{xfrac}

\usepackage{tikz}
\usepackage[mathlines]{lineno}
\usepackage{booktabs}
\usepackage[shortlabels]{enumitem}

\usepackage{hyperref}
\usepackage{cleveref}

\usepackage{lmodern}

% >>>

% <<< formatting

% set default page margins
\makeatletter
\@ifundefined{KOMAoptions}{}{
\KOMAoptions{
    DIV=14,
    parskip=half*,
    overfullrule}}

% custom headers and footers
\pagestyle{plain}
\clearscrheadfoot
\refoot[\pagemark]{\pagemark}
\rofoot[\pagemark]{\pagemark}

% less jarring links
\hypersetup{
    colorlinks=true,
    urlcolor=blue,
    citecolor=black,
    linkcolor=black
}

% simple todo command
\usepackage{todonotes}
\newcommand{\unsure}[1]{
    \todo[linecolor=red,backgroundcolor=red!25,bordercolor=red,inline]{#1}}

% add line numbers
\renewcommand{\linenumberfont}{
    \normalfont\footnotesize\textcolor{black!50}
}

\linenumbers

% number equations within sections
\numberwithin{equation}{section}

% >>>

% <<< commands

% <<< math commands

% integral element with proper spacing
\NewDocumentCommand \dx { O{x} } {\,\mathrm{d} #1}
% style for displaying vectors
\NewDocumentCommand \vect { m } { \mathbold{#1} }

% jump notation
\NewDocumentCommand \jump { sm } {
    \IfBooleanTF#1
    {\left\llbracket #2 \right\rrbracket}
    {\llbracket #2 \rrbracket}
}
% average notation
\NewDocumentCommand \avg { sm } {
    \IfBooleanTF#1
    {\left\langle #2 \right\rangle}
    {\langle #2 \rangle}
}
% inner product
\NewDocumentCommand \ip { m } { \avg{ #1 } }

% ordinary derivative
\NewDocumentCommand \od { m m } { \dfrac{\mathrm{d} #1}{\mathrm{d} #2} }
% material derivative
\NewDocumentCommand \md { m m } { \dfrac{\mathrm{D} #1}{\mathrm{D} #2} }
% partial derivative
\NewDocumentCommand \pd { m m } { \dfrac{\partial #1}{\partial #2} }

% set of real number
\NewDocumentCommand \R {} { \mathbb{R} }

% >>>

% <<< math operators

\DeclareMathOperator{\tr}{tr}
\DeclareMathOperator{\sech}{sech}
\DeclareMathOperator{\argmin}{\operatorname{arg}\,\operatorname{min}}
\DeclareMathOperator{\esssup}{\operatorname{ess}\,\operatorname{sup}}

% >>>

% <<< theorem environments

\newtheorem{theorem}{Theorem}
\newtheorem{remark}{Remark}
\newcommand{\remarkautorefname}{Remark}
\newtheorem{lemma}{Lemma}
\newcommand{\lemmaautorefname}{Lemma}
\newtheorem{example}{Example}
\newcommand{\exampleautorefname}{Example}
\newtheorem{proposition}{Proposition}
\newcommand{\propositionautorefname}{Proposition}
\newtheorem{definition}{Definition}
\newcommand{\definitionautorefname}{Definition}

% >>>

% >>>

% <<< title

\newcommand{\horrule}[1]{\rule{\linewidth}{#1}}
\makeatletter
\def\@maketitle{
\begin{center}
    \normalfont \normalsize

    \horrule{0.5pt}
    \vspace{0.2cm}

    {\huge \bfseries \@title \par}

    \vspace{0.3cm}
    \horrule{2pt}
\end{center}
}
\makeatother

% >>>

% vim:foldmarker=<<<,>>>:foldmethod=marker:nospell


\usepackage{float}
\usepackage{pgfplots}
\pgfplotsset{width=10cm,compat=1.9}

\title{SS WENO Scheme: Implementation Details}

\begin{document}

\maketitle

\section{Introduction} % {{{

We describe here the SS WENO 2-4-2 scheme of~\cite{Fisher2013} as it applies to the
scalar inviscid Burgers equation. The complete system is given by
\[
\begin{cases}
    u_t + f(u)_x = 0, & \quad (t, x) \in [0, T] \times [a, b], \\
    u(0, x) = u_0(x), & \quad x \in [a, b], \\
    u(t, a) = g_a(t), & \quad t \in [0, T] \text{ and } u(t, a) > 0, \\
    u(t, b) = g_b(t), & \quad t \in [0, T] \text{ and } u(t, b) < 0, \\
\end{cases}
\]
where $u_0$ is the initial condition and $(g_a, g_b)$ are boundary conditions
that only apply at inflow boundaries. The analysis of the SS WENO scheme is
also presented in~\cite{Fisher2012} in a bit more detail. We will follow here
the description from the PhD thesis~\cite{Fisher2012}.

\begin{remark}
To ensure numerical entropy stability, the alternative boundary conditions are
considered (see~\cite[Section 4.1]{Fisher2013})
\[
\left\{
\begin{aligned}
\frac{u(t, a) + |u(t, a)|}{3} u(t, a) - g_a(t) =\,\, 0, \\
\frac{u(t, b) - |u(t, a)|}{3} u(t, a) + g_b(t) =\,\, 0,
\end{aligned}
\right.
\]
which are clearly not equivalent to the standard choice of Dirichlet boundary
conditions from above.
\end{remark}

As a general description, the SS WENO 2-4-2 scheme is a fourth-order interior
and second-order boundary WENO finite difference scheme that satisfies entropy
and energy estimates. For this, a special limiter for the fourth-order WENO
reconstruction is given and specific SAT boundary conditions are constructed.
They are all  described below.

% }}}

\section{SS WENO Scheme} % {{{
\label{sc:ssweno}

\begin{figure}[H]
\centering
\begin{tikzpicture}[scale=1.5]
\draw[thick] (0, 0) -- (10, 0);

\draw[thick] (0, -0.3) -- (0, 0.3);
\draw[thick] (10, -0.3) -- (10, 0.3);

% solution points
\foreach \x in {0, 1, ..., 10} {
    \draw (\x, -0.2) -- (\x, 0.2);
    \fill (\x, 0) circle [radius=0.08];
}
\node at (0, -0.3) [below] {$x_0$};
\node at (1, -0.3) [below] {$x_1$};
\node at (2.5, -0.3) [below] {$\cdots$};
\node at (4, -0.3) [below] {$x_{i - 2}$};
\node at (5, -0.3) [below] {$x_{i - 1}$};
\node at (6, -0.3) [below] {$x_{i}$};
\node at (7, -0.3) [below] {$x_{i + 1}$};
\node at (8.5, -0.3) [below] {$\cdots$};
\node at (10, -0.3) [below] {$x_{N - 1}$};

% flux points
\foreach \x in {0.0, 0.3, 1.6, 2.4, 3.5, 4.5, ..., 6.5, 7.6, 8.4, 9.7, 10} {
    \fill[red!70] (\x - 0.05, -0.05) rectangle (\x + 0.05, 0.05);
}
\node at (0, 0) [left] {$\bar{x}_0$};
\node at (0.3, 0) [above] {$\bar{x}_1$};
\node at (4.5, 0) [above] {$\bar{x}_{i - 1}$};
\node at (5.5, 0) [above] {$\bar{x}_{i}$};
\node at (6.5, 0) [above] {$\bar{x}_{i + 1}$};
\node at (9.7, 0) [above] {$\bar{x}_{N - 1}$};
\node at (10, 0) [right] {$\bar{x}_N$};
\end{tikzpicture}
\caption{
    Dual computational grid for the solution points (black circles) and flux points
    (orange squares). We have $N$ grid points $x_i$, for $i \in \{0, \dots, N - 1\}$,
    and $N + 1$ flux points $\bar{x}_i$, for $i \in \{0, \dots, N\}$. The first and
    the last flux points match the solution points.}
\label{fig:weno:grid}
\end{figure}

To put the SS-WENO scheme together using~\cite[Equation 3.42]{Fisher2013}, we
make use of a Method-of-Lines approach. We employ the special dual grid
from~\cite{Fisher2011}, as seen in~\Cref{fig:weno:grid}. We introduce a slight
change in indexing to better match a numerical implementation, i.e. both arrays
are 0-indexed. In the following, the solution is denoted by $u_i \triangleq u(x_i)$.
The fluxes are denoted by $f_i \triangleq f(u(x_i))$ and the reconstructed
fluxes are $\bar{f}_i \approx f(u(\bar{x}_i))$, with the convention that values
at flux points have a bar over them.

In this notation, we write the semi-discrete equation as
\[
\od{u_i}{t} =
-\frac{1}{P_{ii}} (\bar{f}^{SSW}_i - \bar{f}^{SSW}_{i - 1})
+ \frac{1}{P_{ii}} g_i,
\]
where $\bar{f}^{SSW}_i$ is the SS-WENO flux between the nodes $i$ and $i - 1$,
$g_i$ are the boundary conditions and $P$ is the SBP norm matrix
(see~\cite[Appendix A]{Fisher2013}). According to~\cite{Fisher2013}, the
flux is written as
\begin{equation} \label{eq:ssweno:flux}
\bar{f}_i^{SSW} \triangleq \bar{f}^{W}_i + \bar{\delta}_i (\bar{f}^{S}_i - \bar{f}^W_i),
\end{equation}
where $\bar{f}^W_i$ is the WENO reconstructed flux from~\Cref{sc:weno} and
$\bar{f}^S_i$ is the entropy conservative flux from~\Cref{sc:entropy}. The
limiter $\bar{\delta}$ is given by $\bar{\delta}_0 = \bar{\delta}_N = 0$ and,
for $i \in \{1, N - 1\}$,
\begin{equation} \label{eq::ssweno:limiter}
\bar{\delta}_i \triangleq
    \frac{\sqrt{\bar{b}_i^2 + c^2} - \bar{b}_i}{\sqrt{\bar{b}_i^2 + c^2}}
    \in [0, 2],
    \quad \text{where }
    \bar{b}_i \triangleq (w_i - w_{i - 1}) (\bar{f}^S_i - \bar{f}^W_i),
\end{equation}
where $c \in \mathbb{R}_+$ is a small positive constant. In~\cite{Fisher2013},
it is taken to be $c = 10^{-12}$. The limiter values are shown
in~\Cref{fig:ssweno:limiter}.

\begin{figure}[H]
\centering
\begin{tikzpicture}[scale=0.9]
\begin{axis}[
    grid=both,
    axis lines=middle,
    minor tick num=5,
    grid style={line width=.1pt, draw=gray!40},
    major grid style={line width=.2pt,draw=gray!50},
    axis line style={latex-latex},
    enlargelimits={abs=0.1},
    xlabel=\(\bar{b}\),
    ylabel=\(\bar{\delta}\)]
\addplot [domain=-1:1, samples=200, color=black, ultra thick]
    {(sqrt(x^2 + 0.0001) - x) / sqrt(x^2 + 0.0001)};
\end{axis}
\end{tikzpicture}
\caption{
    Graph of the $\bar{\delta}$ limiter for values of $\bar{b}$
    for a value of $c = 10^{-2}$. The profile becomes sharper for smaller
    values of $c$.}
\label{fig:ssweno:limiter}
\end{figure}

The boundary conditions are given by~\cite[Equation 4.8]{Fisher2013}
\begin{equation} \label{eq:ssweno:bc}
\vect{g} = -\vect{e}_0 \left(
\frac{u_0 + |u_0|}{3} u_0 - g_a
\right) + \vect{e}_{N - 1} \left(
\frac{u_{N - 1} - |u_{N - 1}|}{3} u_{N - 1} + g_b
\right),
\end{equation}
which also apply the required upwinding, i.e. only inflow boundaries should be
imposed.

\subsection{Entropy Stability}
\label{ssc:ssweno:stability}

To finish, we give below a short proof that the SS-WENO flux limiter respects
the entropy stability requirement~\cite[Equation 3.41]{Fisher2013}
\[
(w_i - w_{i - 1}) (\bar{f}_i - \bar{f}_i^{S}) \le 0,
\]
where $\bar{f}_i^{S}$ is the entropy conservative flux of~\Cref{sc:entropy}.

\begin{proof}
Replacing the SS-WENO flux~\eqref{eq:ssweno:flux} in the inequality gives
\[
(1 - \bar{\delta}_i) (w_i - w_{i - 1}) (\bar{f}_i^W - \bar{f}_i^{S}) \le 0,
\]
where we can further expand $\bar{\delta}_i$ to obtain
\[
\frac{\bar{b}_i}{\sqrt{\bar{b}_i^2 + c^2}}
    (w_i - w_{i - 1}) (\bar{f}_i^W - \bar{f}_i^{S}) \le 0.
\]

As the numerator is positive and never zero due to $0 < c \ll 1$, we get that
\[
\begin{aligned}
&
\bar{b}_i (w_i - w_{i - 1}) (\bar{f}_i^W - \bar{f}_i^{S}) \le 0 \\
\Leftrightarrow &
-(w_i - w_{i - 1})^2 (\bar{f}_i^W - \bar{f}_i^{S})^2 \le 0,
\end{aligned}
\]
which is true by definition. We note that in~\cite{Fisher2012}, the authors
define
\[
\bar{\delta}_i \triangleq
    \frac{\sqrt{\bar{b}_i^2 + c^2} - \bar{b}_i}{2 \sqrt{\bar{b}_i^2 + c^2}}
    \in [0, 1],
\]
for which the entropy stability requirement is also satisfied.
\end{proof}

% }}}

\section{Fourth Order WENO Construction} % {{{
\label{sc:weno}

We start by looking at a WENO method that results in a fourth-order reconstruction
of the flux. While the SS WENO scheme is described in~\cite{Fisher2013}, the
WENO reconstruction is taken from previous work in~\cite{Fisher2011}.

\begin{figure}[H]
\centering
\begin{tikzpicture}[scale=1.5]
\draw[thick] (2.5, 0) -- (8.5, 0);
\node at (2.5, 0) [left] {$\cdots$};
\node at (8.5, 0) [right] {$\cdots$};

% solution points
\foreach \x in {3, 4, ..., 8} {
    \fill (\x, 0) circle [radius=0.08];
}
\node at (3, -0.3) [below] {$x_{i - 3}$};
\node at (4, -0.3) [below] {$x_{i - 2}$};
\node at (5, -0.3) [below] {$x_{i - 1}$};
\node at (6, -0.3) [below] {$x_{i}$};
\node at (7, -0.3) [below] {$x_{i + 1}$};
\node at (8, -0.3) [below] {$x_{i + 2}$};

% flux points
\foreach \x in {3.5, 4.5, ..., 7.5} {
    \fill[red!70] (\x - 0.05, -0.05) rectangle (\x + 0.05, 0.05);
}
\node at (5.5, 0) [above] {$\bar{x}_{i}$};

% stencils
\draw [thick, dashed] (3, 0) -- (3.5, 1) node [above] {$S^{LL}_i$} -- (4, 0);
\draw [thick, dashed] (4, 0) -- (4.5, 1) node [above] {$S^L_i$} -- (5, 0);
\draw [thick, dashed] (5, 0) -- (5.5, 1) node [above] {$S^C_i$} -- (6, 0);
\draw [thick, dashed] (6, 0) -- (6.5, 1) node [above] {$S^R_i$} -- (7, 0);
\draw [thick, dashed] (7, 0) -- (7.5, 1) node [above] {$S^{RR}_i$} -- (8, 0);
\end{tikzpicture}
\caption{
    Fourth-order WENO candidate stencils for reconstruction at $\bar{x}_i$.
    The stencil $S^{LL}_i$ is valid for $i \in \{3, N - 1\}$, $S^L_i$ is
    valid on $i \in \{2, N - 1\}$, $S^C_i$ is valid on $i \in \{1, N - 1\}$,
    $S^R_i$ is valid on $i \in \{1, N - 2\}$ and $S^{RR}_i$ is valid on
    $i \in \{1, N - 3\}$.}
\label{fig:weno:stencils}
\end{figure}

For this fourth-order WENO reconstruction, we have a five candidate stencils, as
seen in~\Cref{fig:weno:stencils}. Note that the $S^{LL}$ and $S^{RR}$ stencils only
apply near the boundaries to ensure the correct order. This is enforced by the
construction, so they can be evaluated everywhere.

The reconstruction used is given by
\begin{equation} \label{eq:weno:reconstruction}
\bar{f}_i = \sum_{k = LL}^{RR} \bar{\omega}_i^{(k)} \bar{f}^{(k)}_i,
\end{equation}
where $\bar{f}^{(k)}_i$ is the candidate interpolation on stencil $k$. It then
suffices to define the non-linear weights $\bar{\omega}^{(k)}$ and the candidate
interpolated values $\bar{f}^{(k)}$ on each candidate stencil. For a general
description of the construction see~\Cref{ax:weno}.

\subsection{WENO Weights} % {{{
\label{ssc:weno:weights}

\begin{definition} \label{def:weno:weights}
The nonlinear weights of the fourth-order WENO reconstruction are given by
\[
\bar{\omega}^{(k)}_i \triangleq \frac{\bar{\alpha}^{(k)}_i}{\displaystyle
    \sum_{\ell = LL}^{RR} \bar{\alpha}^{(\ell)}_i},
\quad \text{where }
\bar{\alpha}^{(k)}_i \triangleq \bar{d}^{(k)}_i \left(
1 + \frac{\bar{\tau}_i}{\bar{\beta}^{(k)}_i + \bar{\epsilon}_i}\right),
\]
where $\bar{\tau}$ and $\bar{\beta}^{(k)}$ are smoothness indicators. The value of
$\epsilon$ is determined based on the application.
\end{definition}

In~\cite{Fisher2011}, the authors recommend taking
\[
\bar{\epsilon}_i \triangleq \Delta x^4 \max (\|u_0\|, \|u_0'\|),
\]
where the norm $\|\cdot\|$ can be taken as the $P$ norm of the accompanying
SBP scheme. The discontinuity points are ignored when computing $u_0'$. Alternatively,
in~\cite{Fisher2012} we have
\[
\bar{\epsilon}_i \triangleq \min (\bar{c}_i \Delta x^4, \Delta x^2),
\quad \text{where }
\bar{c}_i \triangleq 1
    + \frac{\|u_{K_i}\|^2}{|K_i| + 1}
    + \frac{\|\bar{\beta}_i\|}{|K_i|},
\]
where $|K_i|$ is the number of points in the stencil corresponding to the flux
point $\bar{x}_i$. In our case, $|K_i| \in \{3, 4\}$ only
(see~\Cref{def:weno:target_weights}).

According to \cite[Appendix A]{Fisher2011}, it is recommended to base the
smoothness indicators on the solution variables and not the fluxes themselves.
The resulting indicators are given below.

\begin{definition} \label{def:weno:smoothness}
The smoothness indicators used in~\Cref{def:weno:weights} are given by
\[
\bar{\tau}_i \triangleq
\begin{cases}
    (u_{i + 1} - 3 u_i + 3 u_{i - 1} - u_{i - 2})^2,
    & \quad i \in \{2, N - 2\}, \\
    (u_{i + 2} - 3 u_{i + 1} + 3 u_i - u_{i - 1})^2,
    & \quad i = 1, \\
    (u_i - 3 u_{i - 1} + 3 u_{i - 2} - u_{i - 3})^2,
    & \quad i = N - 1,
\end{cases}
\]
and
\[
\begin{aligned}
    \bar{\beta}_i^{LL} \triangleq\,\, &
    (u_{i - 2} - u_{i - 3})^2, & \qquad i \in \{3, N - 1\}, \\
    \bar{\beta}_i^{L} \triangleq\,\, &
    (u_{i - 1} - u_{i - 2})^2, & \qquad i \in \{2, N - 1\}, \\
    \bar{\beta}_i^{C} \triangleq\,\, &
    (u_i - u_{i - 1})^2, & \qquad i \in \{1, N - 1\}, \\
    \bar{\beta}_i^{R} \triangleq\,\, &
    (u_{i + 1} - u_i)^2, & \qquad i \in \{1, N - 2\}, \\
    \bar{\beta}_i^{RR} \triangleq\,\, &
    (u_{i + 2} - u_{i + 1})^2, & \qquad i \in \{1, N - 3\}.
\end{aligned}
\]
\end{definition}

\begin{remark}
    The smoothness indicators $\bar{\beta}^{(k)}$ can be left undefined outside
    of the valid index ranges, since they will be multiplied by $\bar{d}^{(k)}_i$,
    which is $0$ there.
\end{remark}

\begin{remark}
    As described in~\cite{Fisher2011}, the $\bar{\beta}^{(k)}$ indicators are
    approximation of the first derivatives, e.g.
    \[
    \bar{\beta}^{(C)}_i =
        \Delta x^2 \left(\od{f}{x}(\bar{x}_i)\right)^2
        + \mathcal{O}(\Delta x^3),
    \]
    and the $\bar{\tau}_i$ are approximations of the third derivative, i.e.
    \[
    \bar{\tau}_i =
        \Delta x^6 \left(\od{^3 f}{x^3}(\bar{x}_i)\right)^2
        + \mathcal{O}(\Delta x^7).
    \]

    This is sufficient to ensure that, for a careful choice of $\epsilon$,
    \[
    \frac{\bar{\tau}_i}{\bar{\epsilon}_i + \bar{\beta}^{k}_i} =
    \mathcal{O}(\Delta x^4),
    \]
    which satisfies the truncation order of the scheme. As known for WENO schemes,
    this fails in the vicinity of critical points, which is where $\bar{\epsilon}_i$
    comes in.
\end{remark}

Finally, the target weights are given by~\cite[Appendix C.3]{Fisher2011} as
$\bar{d} \in \mathbb{R}^{N + 1 \times 5}$, where each column corresponds to one
of the candidate stencils. We then make the abuse of notation that
\[
\bar{d}^{(k)}_i \equiv \bar{d}_{i, k}.
\]

\begin{definition} \label{def:weno:target_weights}
The target weights of the fourth-order WENO reconstruction are given by
\[
\bar{d} =
\begin{bmatrix}
0 & 0 & 1 & 0 & 0 \\[1em]
0 & 0 & \dfrac{24}{31} & \dfrac{1013}{4898} & \dfrac{3}{158} \\[1em]
0 & \dfrac{11}{56} & \dfrac{51}{70} & \dfrac{3}{40} & 0 \\[1em]
\dfrac{3}{142} & \dfrac{357}{3266} & \dfrac{408}{575} & \dfrac{4}{25} & 0 \\[1em]
0 & \dfrac{1}{6} & \dfrac{2}{3} & \dfrac{1}{6} & 0 \\[1em]
\vdots & \vdots & \vdots & \vdots & \vdots \\[1em]
\end{bmatrix},
\]
where the columns represent the 5 stencils. The target weights have the symmetry
property
\[
\bar{d}_{i, k} = \bar{d}_{N - i, 4 - k}.
\]
\end{definition}

We can see from~\Cref{def:weno:target_weights} how the candidate stencils vanish
outside of the valid index ranges. For example, the first column corresponding
to $S^{LL}$ is zero except at $i = 3$, i.e. $d_{3, 0} \ne 0$ and, by symmetry,
$d_{N - 3, 4} \ne 0$.

Finally, to ensure the stability of the scheme, the smoothness indicators
require a modification. As described in~\cite{Fisher2011}, we must redefine
the downwind indicator to ensure that it has a smaller weight in the final
reconstruction. The downwind stencil is given by
\[
\bar{D}_i \triangleq
\begin{cases}
    C \text{ if } \bar{u}_i \le 0 \text{ else } RR, &
        \quad i = 1, \\
    L \text{ if } \bar{u}_i \le 0 \text{ else } R, &
        \quad i = 2, \\
    LL \text{ if } \bar{u}_i \le 0 \text{ else } R, &
        \quad i = 3, \\
    L \text{ if } \bar{u}_i \le 0 \text{ else } R, &
        \quad i \in \{4, N - 4\}, \\
    L \text{ if } \bar{u}_i \le 0 \text{ else } RR, &
        \quad i = N - 3, \\
    L \text{ if } \bar{u}_i \le 0 \text{ else } R, &
        \quad i = N - 2, \\
    LL \text{ if } \bar{u}_i \le 0 \text{ else } C, &
        \quad i = N - 1, \\
\end{cases}
\]
where $\bar{u}_i \triangleq (u_i + u_{i - 1}) / 2$. Then, we set
\[
\bar{\beta}^{\bar{D}_i}_i =
\left(
    \frac{1}{|K_i|}
    \sum_{k \in K_i} \left[\bar{\beta}^{(k)}_i\right]^p
\right)^{\frac{1}{p}},
\]
where $p \in \mathbb{N}$ even. In~\cite{Fisher2012}, this is taken to be $p = 4$.
The general intuition being that taking $p \to \infty$ will yield the $\ell^\infty$
norm. Modifying the smoothness indicator in this way, i.e. such that the downwind
indicator is larger than the others, will ensure that the corresponding
$\omega^{\bar{D}_i}_i$ will be smaller and the scheme will not select the
downwind stencil. This adds additional dissipation and stabilizes the standard
WENO scheme. Note that the energy stable WENO schemes from~\cite{Yamaleev2009}
do not require such changes.

\begin{remark}
To follow the text of the method, i.e. to ensure that $\bar{\beta}_i^{\bar{D}_i}$
is larger than its colleagues, we must set
\[
\bar{\beta}^{\bar{D}_i}_i =
\left(
    \sum_{k \in K_i} \left[\bar{\beta}^{(k)}_i\right]^p
\right)^{\frac{1}{p}}
\]
without the averaging. The original definition is only accurate in the limit.
\end{remark}

% }}}

\subsection{WENO Interpolation} % {{{
\label{ssc:weno:interp}

We now require a definition of the interpolated values $\bar{f}^{(k)}$ to
complete the WENO reconstruction. The required interpolating functions are
described in~\cite[Appendix C.2]{Fisher2011} and~\cite[Equation A.15]{Fisher2012}.
In general, we write
\[
\bar{f}_i^{(k)} \triangleq \sum_{j = 0}^{p - 1} \bar{h}^{(k)}_{ij} f(u_{i + k + j - p - 1}),
\]
where $p$ is the number of points in the stencils and $\bar{h}^{(k)}$ are
coefficient matrices. They are given below explicitly for the case of $p = 2$,
so we write
\[
\begin{aligned}
\bar{f}^{LL}_i \triangleq\,\, &
    \bar{a}^{LL}_i f(u_{i - 3}) + \bar{b}^{LL}_i f(u_{i - 2}), &
    \quad i \in \{3, N - 1\}, \\
\bar{f}^{L}_i \triangleq\,\, &
    \bar{a}^{L}_i f(u_{i - 2}) + \bar{b}^{L}_i f(u_{i - 1}), &
    \quad i \in \{2, N - 1\}, \\
\bar{f}^{C}_i \triangleq\,\, &
    \bar{a}^{C}_i f(u_{i - 1}) + \bar{b}^{C}_i f(u_i), &
    \quad i \in \{1, N - 1\}, \\
\bar{f}^{R}_i \triangleq\,\, &
    \bar{a}^{R}_i f(u_i) + \bar{b}^{R}_i f(u_{i + 1}), &
    \quad i \in \{1, N - 2\}, \\
\bar{f}^{RR}_i \triangleq\,\, &
    \bar{a}^{RR}_i f(u_{i + 1}) + \bar{b}^{RR}_i f(u_{i + 2}), &
    \quad i \in \{1, N - 3\}, \\
\end{aligned}
\]
where $\bar{a}^{(k)}_i \triangleq \bar{h}^{(k)}_{i, 0}$ and
$\bar{b}^{(k)}_i \triangleq \bar{h}^{(k)}_{i, 1}$. Note that the valid index
ranges match those of the $\bar{\beta}$ smoothness indicators from
\Cref{def:weno:smoothness}.

\begin{definition} \label{def:weno:interp}
The interpolation coefficients $\bar{h}^{(k)} \in \mathbb{R}^{N + 1 \times 2}$
are given by
\begin{gather}
\bar{h}^{LL} \triangleq
\begin{bmatrix}
0 & 0 \\[1em]
0 & 0 \\[1em]
0 & 0 \\[1em]
-\dfrac{71}{48} & \dfrac{119}{48} \\[1em]
0 & 0 \\[1em]
\vdots & \vdots
\end{bmatrix},
\qquad
\bar{h}^{RR} \triangleq
\begin{bmatrix}
0 & 0 \\[1em]
\dfrac{127}{48} & -\dfrac{79}{48} \\[1em]
\dfrac{29}{12} & -\dfrac{17}{12} \\[1em]
\dfrac{121}{48} & -\dfrac{73}{48} \\[1em]
0 & 0 \\[1em]
\vdots & \vdots
\end{bmatrix}, \\
\bar{h}^L \triangleq
\begin{bmatrix}
0 & 0 \\[1em]
0 & 0 \\[1em]
-\dfrac{7}{12} & \dfrac{19}{12} \\[1em]
-\dfrac{23}{48} & \dfrac{71}{48} \\[1em]
-\dfrac{1}{2} & \dfrac{3}{2} \\[1em]
\vdots & \vdots
\end{bmatrix},
\qquad
\bar{h}^C \triangleq
\begin{bmatrix}
1 & 0 \\[1em]
\dfrac{31}{48} & \dfrac{17}{48} \\[1em]
\dfrac{5}{12} & \dfrac{7}{12} \\[1em]
\dfrac{25}{48} & \dfrac{23}{48} \\[1em]
\dfrac{1}{2} & \dfrac{1}{2} \\[1em]
\vdots & \vdots
\end{bmatrix},
\qquad
\bar{h}^R \triangleq
\begin{bmatrix}
0 & 0 \\[1em]
\dfrac{79}{48} & -\dfrac{31}{48} \\[1em]
\dfrac{17}{12} & -\dfrac{5}{12} \\[1em]
\dfrac{73}{48} & -\dfrac{25}{48} \\[1em]
\dfrac{3}{2} & -\dfrac{1}{2} \\[1em]
\vdots & \vdots
\end{bmatrix},
\end{gather}
and satisfy the symmetry condition
\[
\bar{h}^{(k)}_{i, j} = \bar{h}^{(4 - k)}_{N - i, p - j - 1}.
\]
\end{definition}

% }}}

\subsection{WENO Flux Splitting} % {{{
\label{ssc:weno:flux}

Using the reconstruction as presented in~\eqref{eq:weno:reconstruction} is
not going to be stable. This is obvious, as it does not perform proper upwinding
To achieve this, a form a flux splitting is used through the decomposition
(see e.g. \cite{Fisher2012})
\[
f = f^+ + f^-,
\]
where
\[
\od{f^+}{u} > 0
\quad \text{and} \quad
\od{f^-}{u} < 0.
\]

The choice of splitting used in~\cite{Fisher2012} is the standard Lax-Friedrichs
splitting
\[
f^\pm_i \triangleq \frac{1}{2} (f_i \pm a_i u_i),
\]
where $a_i$ is a measure of the flux Jacobian eigenvalue magnitude at the point
$x_i$. The common choice is to take $a_i \triangleq |u_i|$ for Burgers' equation. In~\cite{Fisher2012}, the authors take
\[
a_i = \max_{j \in K_i} |u_j|,
\]
for all the points $j \in K_i$ the stencil used to reconstruct the flux. To
tie the flux splitting into the WENO reconstruction, we say that
\[
\begin{aligned}
\bar{f}^{(k), +}_i =\,\, & \sum_{k = LL}^{RR} \omega^{(k), +}_i \bar{f}^+_i, \\
\bar{f}^{(k), -}_i =\,\, & \sum_{k = LL}^{RR} \omega^{(k), -}_i \bar{f}^-_i,
\end{aligned}
\]
where the downwind stencil for the $f^-$ reconstruction is obviously mirrored
from the one in~\Cref{ssc:weno:weights}.

% }}}

% }}}

\section{Entropy Conservative Flux} % {{{
\label{sc:entropy}

In this section we will look at the high-order extension of the of the entropy
conservative flux from, as described in~\cite{Fisher2013}. The second-order
expression for an entropy conservative flux is given in~\Cref{ax:entropy} by
\[
\bar{f}^s(u_i, u_j) \triangleq
    \frac{1}{6} (u_i u_j + u_i u_j + u_j u_j).
\]

In~\cite{Fisher2013}, a high-order extension to finite domains is proposed.
It is given by
\[
\bar{f}^{S}_i \triangleq \sum_{k = i}^{N - 1} \sum_{\ell = 0}^{i - 1}
2 Q_{\ell, k} \bar{f}^s(u_\ell, u_k),
\]
for $i \in \{1, N - 1\}$. Here, $Q \in \mathbb{R}^{N \times N}$ is the SBP
matrix used to construct the first-order derivative. In the fourth-order case,
it is given by~\cite[Equation A.2]{Fisher2013}
\[
Q =
\begin{bmatrix}
    -\dfrac{1}{2} & \dfrac{59}{96} & -\dfrac{1}{12} & -\dfrac{1}{32}
    & 0 & 0 & 0 & 0 & 0 & \cdots \\[1em]
    -\dfrac{59}{96} & 0 & \dfrac{59}{96}
    & 0 & 0 & 0 & 0 & 0 & 0 & \cdots \\[1em]
    \dfrac{1}{12} & -\dfrac{59}{96} & 0 & \dfrac{59}{96} & -\dfrac{1}{12}
    & 0 & 0 & 0 & 0 & \cdots \\[1em]
    \dfrac{1}{32} & 0 & -\dfrac{59}{96} & 0 & \dfrac{2}{3} & -\dfrac{1}{12}
    & 0 & 0 & 0 & \cdots \\[1em]
    0 & 0 & \dfrac{1}{12} & -\dfrac{2}{3} & 0 & \dfrac{2}{3} & -\dfrac{1}{12}
    & 0 & 0 & \vdots \\
    \vdots & \vdots & \vdots & \vdots & \vdots & \vdots & \vdots & \vdots & \vdots
    & \ddots \\[1em]
\end{bmatrix}
\]
and satisfies the symmetry condition
\[
Q_{ij} = -Q_{N - i, N - j}.
\]

By the construction of the high-order entropy flux, we can see that only
rows $< i$ and columns $\ge i$ are used in the construction. Therefore, for
interior points, where $i \in \{4, \dots, N - 4\}$, we have
\[
\begin{aligned}
\bar{f}^S_i \triangleq\,\, &
2 Q_{i, i + 1} f^s(u_{i - 1}, u_i)
+ 2 Q_{i, i + 2} f^s(u_{i - 1}, u_{i + 1})
+ 2 Q_{i - 1, i + 1} f^s(u_{i - 2}, u_i)) \\
=\,\, &
\frac{4}{3} f^s(u_{i - 1}, u_i)
- \frac{1}{6} (f^s(u_{i - 1}, u_{i + 1}) + f^s(u_{i - 2}, u_i)).
\end{aligned}
\]

The boundary expressions are more complicated, but can be constructed
explicitly from the initial definition. They are
\[
\left\{
\begin{aligned}
f^S_i \triangleq \,\, & f(u_0), \\
f^S_i \triangleq \,\, &
    2 Q_{i - 1, i} f^s(u_{i - 1}, u_i)
    - \frac{1}{6} f^s(u_0, u_2)
    - \frac{1}{16} f^s(u_0, u_3), \\
f^S_2 \triangleq \,\, &
\frac{59}{24} f^s(u_1, u_2)
- \frac{1}{6} f^s(u_0, u_2)
- \frac{1}{16} f^s(u_0, u_3), \\
f^S_3 \triangleq \,\, &
\frac{59}{48} f^s(u_2, u_3)
- \frac{1}{16} f^s(u_0, u_3)
- \frac{1}{6} f^s(u_2, u_4),
\end{aligned}
\right.
\]
while the right boundary is flipped and gives
\[
\begin{aligned}
f^S_N = \,\, & f(u_N), \\
f^S_{N - 1} \triangleq \,\, &
    -\frac{59}{48} f^s(u_{N - 1}, u_{N - 2})
    + \frac{1}{6} f^s(u_{N - 1}, u_{N - 3})
    + \frac{1}{12} f^s(u_{N - 1}, u_{N - 4}), \\
f^S_{N - 2} \triangleq \,\, &
    -\frac{59}{48} f^s(u_2, u_3)
    + \frac{1}{6} f^s(u_2, u_4)
    + \frac{1}{12} f^s(u_0, u_3), \\
f^S_{N - 3} \triangleq \,\, &
    -\frac{59}{48} f^s(u_3, u_4)
    + \frac{1}{16} f^s(u_3, u_5)
    + \frac{1}{6} f^s(u_2, u_4).
\end{aligned}
\]

% }}}

\appendix
\section{Finite Difference WENO Reconstruction} % {{{
\label{ax:weno}

For completeness, we give here a short description of how WENO reconstruction is
performed in the context of finite difference methods. The text generally
follows the well-known presentation from~\cite{Shu2009}. In the case of finite
difference WENO reconstructions, we are given the solution point values $u_i$ and
$f(u_i)$ and are expected to reconstruct the value of the flux $\bar{f}_i$.

We start by constructing a function $h(x)$ such that (see~\cite[Section 3.2]{Shu2009})
\begin{equation} \label{eq:ax:weno_average}
\frac{1}{\Delta x} \int_{x - \frac{\Delta x}{2}}^{x + \frac{\Delta x}{2}}
    h(\xi) \dx[\xi] = f(u(x)),
\end{equation}
where $\Delta x$ is the uniform grid spacing. It is clear from this construction
that the average of $h$ on the dual $\bar{x}_i$ grid (see~\Cref{fig:weno:grid}) satisfies
\[
\hat{h}_i = f(u_i),
\]
with the caveat that the cell is not always centred around $x_i$ (this is only
true for interior points). By taking a derivative of~\eqref{eq:ax:weno_average}
we get
\[
\frac{1}{\Delta x} (h(\bar{x}_{i + 1}) - h(\bar{x}_i)) = f(u)_x\Big|_{x = x_i},
\]
which implies that we can take $\bar{f}_i \equiv h(\bar{x}_i)$ as our flux
approximation. Therefore, we are left with reconstructing $h(\bar{x}_i)$ from
the cell-averaged values $\hat{h}_i$, completely equivalently to the finite
volume case.

Then, to reconstruct $h(\bar{x}_i)$ we follow~\cite[Section 2.2]{Shu2009}.
We start by defining a polynomial $p^C(x)$ which interpolates $\hat{h}$ on $S^C$,
i.e.
\[
\begin{aligned}
p^C(x_{i - 1}) =\,\, & a_1^C x_{i - 1} + a_0^C = \hat{h}_{i - 1}, \\
p^C(x_i) =\,\, & a_1^C x_i + a_0^C = \hat{h}_i,
\end{aligned}
\]
and evaluate $p^C(\bar{x}_i) = \bar{h}^{C}_i$. Some simple algebra shows that
\[
\bar{h}^C_i = \frac{1}{2} \hat{h}_i + \frac{1}{2} \hat{h}_{i - 1},
\]
or in terms of the flux
\[
\bar{f}^C_i = \frac{1}{2} f_i + \frac{1}{2} f_{i - 1}
= f(\bar{x}_i) + \mathcal{O}(\Delta x^2),
\]
which matches the previous central interior stencil from~\Cref{def:weno:interp}.
The remaining interior points can be reconstructed in an equivalent manner.
For the boundary points, the construction is similar, but it is more difficult
to define the points $\bar{x}_i$, since they are no longer at the midpoint of
$[x_{i - 1}, x_i]$.

Finally, once we have the required stencil interpolation we can combine the
lower-order interpolation to obtain a high-order result. For example, if we
consider $\{S^L, S^C, S^R\}$, then we wish to find $\{\gamma^L, \gamma^C, \gamma^R\}$
such that
\[
\bar{f}_i = \gamma^L \bar{f}^L_i + \gamma^C \bar{f}^C_i + \gamma^R \bar{f}_i
\]
is a fourth-order approximation. For example, a fourth-order Lagrange interpolant
is given by
\[
\bar{f}_i = -\frac{1}{12} f_{i - 2} + \frac{7}{12} f_{i - 1} + \frac{7}{12} f_i - \frac{1}{12} f_{i + 1},
\]
which implies that
\[
\gamma^L = \frac{1}{6},
\gamma^C = \frac{2}{3}
\quad \text{and }
\gamma^R = \frac{1}{6}.
\]

\begin{proof}
From the two-point stencil reconstructions, we have that
\[
\bar{f}_i =
\gamma^L \left(-\frac{1}{2} f_{i - 2} + \frac{3}{2} f_{i - 1}\right)
+ \gamma^C \left(\frac{1}{2} f_{i - 1} + \frac{1}{2} f_i\right)
+ \gamma^R \left(\frac{3}{2} f_i - \frac{1}{2} f_{i + 1}\right),
\]
which needs to match the fourth-order expansion above. We are left with a system
of 4 equations and 3 unknowns. From the coefficients of $f_{i - 2}$ and $f_{i + 1}$,
we directly have that
\[
\begin{aligned}
-\frac{1}{2} \gamma^L =\,\, -\frac{1}{12} \implies \gamma^L = \frac{1}{6}, \\
-\frac{1}{2} \gamma^R =\,\, -\frac{1}{12} \implies \gamma^R = \frac{1}{6}.
\end{aligned}
\]

Then, from the coefficient of $f_{i - 1}$, we have that
\[
\frac{1}{2} (3 \gamma^L + \gamma^C) = \frac{7}{12} \implies
\gamma^C = \frac{2}{3}.
\]
\end{proof}

% }}}

\section{Entropy Conservative Flux} % {{{
\label{ax:entropy}

Entropy stable fluxes have been studied for a while now, for example
in~\cite{Tadmor2003} or~\cite{Ismail2009}. We will give here a short introduction
relevant to the scalar Burgers equation case we are working with. In general,
the continuous solution must satisfy
\[
\pd{S}{t} + \pd{F}{x} \le 0
\quad \Leftrightarrow \quad
\od{}{t} \int_a^b S + F\Big|_{x = a}^{x = b} \le 0,
\]
where the weak form is satisfied by weak solutions and strong solutions alike.
In the above, $S(u)$ is called the entropy and $F(u)$ is the entropy flux. Note
that the entropy pair $(S, F)$ is not guaranteed to exist. They must satisfy the
relation
\[
\od{S}{u} \od{f}{u} = \od{F}{u},
\]
where $f$ is the flux of the original conservation law. The entropy must also
be convex, so its Hessian is positive definite, i.e.
\[
v \od{^2S}{u^2} v > 0,
\]
for all smooth functions $v \ne 0$. Due to these properties, it yields a one-to-one
mapping from conserved variables to entropy variables $w = S_u$ and the original
inequality. For physical purposes, it is important to note that entropy must
decrease across a shock. Due to the symmetry of the entropy function construction,
we can also define a potential $\varphi$ and potential flux $\psi$ as
\[
\varphi \triangleq w u - S
\quad \text{and} \quad
\psi \triangleq w f - F,
\]
such that $u = \varphi_w$ and $f = \psi_w$. Due to the construction, the potential
function is also convex with respect to the entropy variables $w$.

For Burgers' equation, we can simply take
\[
(S, F) = \left(\frac{u^2}{2}, \frac{u^3}{3}\right)
\quad \text{and} \quad
(\phi, \psi) = \left(\frac{w^2}{2}, \frac{w^3}{6}\right),
\]
such that the entropy variable is $w = S_u = u$. This is not sufficient at the
continuous level, as the entropy equation must hold for all admissible entropy
pairs, but it is a good starting point for designing numerical methods.

For a scalar conservation law, we consider the semi-discrete form
\[
\od{u_i}{t} = -\frac{1}{P_{ii}} (\bar{f}_{i + 1} - \bar{f}_i) + \frac{1}{P_ii} g_i
\quad \Leftrightarrow \quad
\od{\vect{u}}{t} = -P^{-1} \Delta \bar{\vect{f}} + P^{-1} \vect{g},
\]
where $P$ is the SBP norm matrix. The flux must be constructed in such a way
that it satisfies entropy inequalities. In~\cite{Fisher2013}, the authors
propose a local condition to ensure entropy consistency
\[
(w_i - w_{i - 1}) \bar{f}^s_i = \psi_i - \psi_{i - 1}
\]
and entropy stability is achieved if the flux satisfies
\[
(w_i - w_{i - 1}) (\bar{f}_i - \bar{f}^s_i) \le 0.
\]

Substituting the potential flux for Burgers' equation, we get that
\[
(w_i - w_{i - 1}) \bar{f}^s_i = \frac{1}{6} (w_{i} - w_{i - 1})
\implies
\bar{f}^s_i =  \frac{1}{6} (u_i u_i + u_i u_{i - 1} + u_{i - 1} u_{i - 1}),
\]
which is a second-order entropy consistent flux. The high-order extension of
these ideas is presented in~\Cref{sc:entropy}. However, the general idea
consists of reformulating the above expression. Namely, we can write
\[
-\frac{1}{P_{ii}} (\bar{f}^S_{i + 1} - \bar{f}^S_i) =
\sum_{j = 0}^N \left(
\frac{1}{2}
\begin{bmatrix}
0 & 1 & 0 & \cdots & -1 \\
-1 & 0 & 1 & \cdots & 0 \\
0 & -1 & 0 & \cdots & 0 \\
\vdots & \vdots & \ddots & & \vdots \\
1 & 0 & \cdots & -1 & 0
\end{bmatrix}
\odot
\begin{bmatrix}
\bar{f}^s(u_0, u_0) & \cdots & \bar{f}^s(u_0, u_N) \\
\vdots & \cdots & \vdots \\
\bar{f}^s(u_N, u_0) & \cdots & \bar{f}^s(u_N, u_N)
\end{bmatrix}
\right)_{ji},
\]
where $\odot$ is the Hadamard product and the entropy conservative flux is given
by
\[
\bar{f}^s(u, v) = \frac{1}{6} (u^2 + uv + v^2).
\]

Then, we can write our semi-discrete equations as
\[
\od{\vect{u}}{t}
+ 2 (\vect{Q} \odot \vect{F}^s) \vect{1}
+ \vect{B} (\vect{f}^S_b - \vect{f}^s) = 0,
\]
where $\vect{B} = \operatorname{diag}(-1, 0, \dots, 1)$ is the standard SBP
boundary operator. As we can see, the original formulation uses a second-order
method to arrive at entropy conservation. However, we can replace $\vect{Q}$
with high-order SBP operators to obtain higher-order entropy conservative
schemes. In particular, the fourth-order scheme is presented in~\Cref{sc:entropy}.

% }}}

\begin{thebibliography}{9}
\bibitem{Tadmor2003}
    E. Tadmor,
    \textit{
        Entropy Stability Theory for Difference Approximations of Nonlinear
        Conservation Laws and Related Time-Dependent Problems},
    Acta Numerica, Vol. 12, pp. 451--512, 2003,
    \url{http://dx.doi.org/10.1017/s0962492902000156}.
\bibitem{Ismail2009}
    F. Ismail, P. L. Roe,
    \textit{
        Affordable, Entropy-Consistent {Euler} Flux Functions II: Entropy
        Production at Shocks},
    Journal of Computational Physics, Vol. 228, pp. 5410--5436, 2009,
    \url{http://dx.doi.org/10.1016/j.jcp.2009.04.021}.
\bibitem{Shu2009}
    C.-W. Shu,
    \textit{
        High-Order Weighted Essentially Non-oscillatory Schemes for Convection
        Dominated Problems},
    {SIAM} Review, Vol. 51, pp. 82--126, 2009,
    \url{http://dx.doi.org/10.1137/070679065}.
\bibitem{Yamaleev2009}
    N. K. Yamaleev, M. H. Carpenter,
    \textit{A Systematic Methodology for Constructing High-Order Energy Stable {WENO} Schemes},
    Journal of Computational Physics, Vol. 228, pp. 4248--4272, 2009,
    \url{http://dx.doi.org/10.1016/j.jcp.2009.03.002}.
\bibitem{Fisher2011}
    T. C. Fisher, M. H. Carpenter, N. K. Yamaleev, S. H. Frankel,
    \textit{
        Boundary Closures for Fourth-Order Energy Stable Weighted Essentially
        Non-Oscillatory Finite-Difference Schemes},
    Journal of Computational Physics, Vol. 230, pp. 3727--3752, 2011,
    \url{http://dx.doi.org/10.1016/j.jcp.2011.01.043}.
\bibitem{Fisher2012}
    T. C. Fisher,
    \textit{
        High-Order L2 Stable Multi-Domain Finite Difference Method for
        Compressible Flows},
    PhD Thesis, 2012,
    \url{https://docs.lib.purdue.edu/dissertations/AAI3544141/}.
\bibitem{Fisher2013}
    T. C. Fisher, M. H. Carpenter,
    \textit{
        High-Order Entropy Stable Finite Difference Schemes for Nonlinear
        Conservation Laws: Finite Domains},
    Journal of Computational Physics, Vol. 252, pp. 518--557, 2013,
    \url{http://dx.doi.org/10.1016/j.jcp.2013.06.014}.
\bibitem{Jiang2013}
    Y. Jiang, C.-W. Shu, M. Zhang,
    \textit{
        An Alternative Formulation of Finite Difference Weighted ENO Schemes
        With Lax-Wendroff Time Discretization for Conservation Laws},
    {SIAM} Journal on Scientific Computing, Vol. 35, pp. A1137--A1160, 2013,
    \url{http://dx.doi.org/10.1137/120889885}.
\end{thebibliography}

\end{document}
